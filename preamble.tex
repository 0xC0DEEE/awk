% vim: ts=4 sts=4 sw=4 et tw=75
% preamble here.

\documentclass[nofonts]{book}

%\usepackage{geometry}
\usepackage{fontspec}
\usepackage{xeCJK}
\usepackage{hyperref}
\usepackage{fancyhdr}
\usepackage{verbatim}

% from package geometry
%\geometry{bottom = 2cm, outer = 2cm}

% page style from package fancyhdr
\pagestyle{fancy}
\fancyhf{}
\fancyhead[EL]{\thepage}
\fancyhead[ER]{\nouppercase{\leftmark}}
\fancyhead[OR]{\thepage}
\fancyhead[OL]{\nouppercase{\rightmark}}

% from package hyperref
\hypersetup{
    bookmarksnumbered = true
}

% from package fontspec and xeCJK
\setCJKmainfont[Scale=1.0]{AR PL UMing CN}
\setCJKsansfont[Scale=1.0]{AR PL UMing CN}
\setCJKmonofont[Scale=1.0]{AR PL UMing CN}
\setmainfont{Liberation Serif}
\setsansfont{FreeSans}
\setmonofont{FreeMono}

% the name of file or directory
\newcommand\filename[1]{\texttt{#1}}

% the content of file
\newenvironment{file}%
{\verbatim}%
{\endverbatim}

% awk program, from package verbatim
\newenvironment{awkcode}%
{\verbatim}%
{\endverbatim}

% shell command, from package verbatim
\newenvironment{shell}%
{\verbatim}%
{\endverbatim}

% pattern for many situations
\newenvironment{pattern}%
{\begin{quotation}}%
{\end{quotation}}

% term in English
\newcommand\term[1]{\textit{#1}}
% term in Chinese
\newcommand\cterm[1]{\textbf{#1}}

% word that shows frequently
\newcommand\awk{\texttt{awk}}
\newcommand\print{\texttt{print}}
\newcommand\nf{\texttt{NF}}

\title{AWK 程序设计语言}
\author{Alfred V.Aho \and Brian W.Kernighan \and Peter J.Weinberger \and
    \url{https://github.com/wuzhouhui/awk}
}
