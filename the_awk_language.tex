% vim: ts=4 sts=4 sw=4 et tw=75

\chapter{Awk 语言}
\label{chap:the_awk_language}

\marginpar{21}
这一章解释---大部分都带有例子---构成一个 awk 程序的构造要素. 由于这次要描述
的是整个语言, 所以材料会非常琐细, 所以我们建议读者只需要浏览一下即可, 当
有必要的时候再回来查看细节.

最简单的 awk 程序是一个由多个 \patact 构成的序列:
\begin{pattern}
\textit{pattern} \texttt{\{} \textit{action} \texttt{\}}
\textit{pattern} \texttt{\{} \textit{action} \texttt{\}}
...
\end{pattern}
在某些语句中, 模式可以不存在; 还有些语句, 动作及其包围它的花括号也可以不存
在. 如果你的程序经过 awk 检查后没有发现语法错误, 它就会每次读取一个输入行,
对读取到的每一行, 按顺序检查每一个模式. 对每一个与当前行匹配的模式, 对应
的动作就会被执行. 一个缺失的模式匹配每一个输入行, 因此每一个不带有模式的动
作对每一个输入行都会执行. 只含有模式而没有动作的语句, 会打印每一个匹配模式
的输入行. 大部分情况下, 在这一章出现的术语 ``输入行'' 与 ``记录'' 被当作
一对同义词. 在 \ref{sec:input} 节, 我们都会讨论多行记录, 多行记录指的是一
个记录包含了多个行.

这一章的第一节会详细描述模式. 第二节通过描述表达式, 赋值语句与流程控制语句,
展开对动作的讨论. 剩下的小节包括函数定义, 输出, 输入, 以及 awk 如果调用其
他程序. 大部分小节都包含对主要性质的小节.

\subsection{输入文件\filename{countries}}
\label{subsec:the_input_file_countries}

作为本章许多 awk 程序的输入数据, 我们将使用文件 \filename{countries}. 每一
行都包括一个国家的名字, 面积 (以千平方英里为单位), 人口 (以百万为单位), 以
及这个国家所在的大陆. 数据来源于 1984 年, 苏联被归到了亚洲. 在文件里, 四列
数据用制表符分隔; 用一个空白符分隔 \texttt{North} (\texttt{South}) 与
\texttt{America}.

文件 \filename{countries} 包含下面几行:
\marginpar{22}
\begin{file}
    USSR        8649    275     Asia
    Canada      3852    25      North America
    China       3705    1032    Asia
    USA         3615    237     North America
    Brazil      3286    134     South America
    India       1267    746     Asia
    Mexico      762     78      North America
    France      211     55      Europe
    Japan       144     120     Asia
    Germany     96      61      Europe
    England     94      56      Europe
\end{file}

在这一章的剩下部分, 如果没有显式给出输入数据, 默认将 \filename{countries}
作为输入数据.
