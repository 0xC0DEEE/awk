\chapter{报表与数据库}
\label{reports_and_databases}

\marginpar{89}
这章展示如何使用 awk 从文件中提取信息, 并生成报表, 我们把重点放在表格数据,
但是同样的技术也可以用在更加复杂的输入格式上. 本章的主题是开发一个可以与
其他程序配合使用的程序. 我们将会看到大量的工作中会经常遇到的数据处理问题,
这些问题很难一步解决, 但是如果多次遍历数据, 就相对比较容易一些.

本章的第一部分讨论如何扫描单个文件来生成报表, 虽然报表的最终格式的确需要花点
心思, 但是其实扫描步骤也是挺复杂的. 第二部分讨论如何从多个相关的文件中
收集数据, 我们考虑用一种比较一般的方法来解决这个问题, 通过把文件组当成
关系数据库, 这样做的好外是字段可以用名字来标识, 而不是数字.

\section{报表生成}
\label{sec:generating_reports}

Awk 可以从文件中挑选数据, 并将挑选到的数据格式化成报表. 我们将使用一个
三步骤过程来生成报表: 准备, 排序, 格式化. 准备步骤包括选择数据, 可能的话
还会对数据进行一些运算, 进而得到期望的信息; 如果我们想让数据按照某种特定
的顺序排列, 就必须使用排序步骤, 排序操作可以通过将准备阶段的输出输送给系统
的排序命令来完成; 格式化操作由第 2 个 awk 程序来完成, 它根据已排序的数据
生成报表. 为了详细说明, 在这一节 我们利用第 \ref{chap:the_awk_language} 
章的文件\filename{countries} 来生成几张报表.

\subsection{一个简单的报表}
\label{subsec:a_simple_report}

假设我们想要一张报表, 这张表包含了每个国家的人口, 面积, 及人口密度. 
我们还希望国家按照所在的大洲进行分组, 大洲按照字母顺序排列, 在每个大洲里,
国家按照人口密度的降序排列, 就像这样:
\marginpar{90}
\begin{shell}
    CONTINENT       COUNTRY    POPULATION    AREA    POP. DEN.
    Asia            Japan          120        144      833.3
    Asia            India          746       1267      588.8
    Asia            China         1032       3705      278.5
    Asia            USSR           275       8649       31.8
    Europe          Germany         61         96      635.4
    Europe          England         56         94      595.7
    Europe          France          55        211      260.7
    North America   Mexico          78        762      102.4
    North America   USA            237       3615       65.6
    North America   Canada          25       3852        6.5
    South America   Brazil         134       3286       40.8
\end{shell}

生成报表的前两个阶段由程序 \verb'prep1' 完成, 当文件 \filename{countries}
作为输入时, \verb'prep1' 提取并计算相关的信息, 并对其进行排序:
\begin{awkcode}
    # prep1 - prepare countries by continent and pop. den.

    BEGIN { FS = "\t" }
          { printf("%s:%s:%d:%d:%.1f\n",
                $4, $1, $3, $2, 1000*$3/$2) | "sort -t: +0 -1 +4rn"
          }
\end{awkcode}
输出是一系列的行, 每行都包括 5 个字段, 用冒号分隔, 从左至右, 字段依次表示
大洲, 国家, 人口, 面积, 以及人口密度:
\begin{shell}
    Asia:Japan:120:144:833.3
    Asia:India:746:1267:588.8
    Asia:China:1032:3705:278.5
    Asia:USSR:275:8649:31.8
    Europe:Germany:61:96:635.4
    Europe:England:56:94:595.7
    Europe:France:55:211:260.7
    North America:Mexico:78:762:102.4
    North America:USA:237:3615:65.6
    North America:Canada:25:3852:6.5
    South America:Brazil:134:3286:40.8
\end{shell}
程序 \verb'prep1' 把输出直接输送给 \verb'sort' 命令, 参数 \verb'-t' 告诉 
\verb'sort' 把冒号作为字段分隔符, \verb'+0 -1' 表示把第1个字段作为排序的
主键, 参数 \verb'+4rn' 表示把第 5 个字段作为次要主键, 按照数值的逆序进行
排序. (在 \ref{sec:a_sort_generator} 节, 我们将展示一个生成排序的程序,
这个程序可以从单词描述中生成排序所需的参数列表)

如果你的系统不支持管道, 那就把 \verb'sort' 命令删除, 使用 \verb'print >'
\textit{file} 直接输出到文件中, 然后再利用单独的步骤对文件排序,
这个方法适用于本章的所有例子.

现在我们已经完成了三个步骤中的前两个: 准备与排序, 现在所要做的是把数据
格式化成我们想要的报表格式, 程序 \verb'form1' 做的正是这个工作:
\marginpar{91}
\begin{awkcode}
    # form1 - format countries data by continent, pop. den.

    BEGIN { FS = ":"
            printf("%-15s %-10s %10s %7s %12s\n",
                "CONTINENT", "COUNTRY", "POPULATION",
                "AREA", "POP. DEN.")
          }
          { printf("%-15s %-10s %7d %10d %10.1f\n",
                $1, $2, $3, $4, $5)
          }
\end{awkcode}
期望中的报表可以通过键入
\begin{shell}
    awk -f prep1 countries | awk -f form1
\end{shell}
来得到.

\verb'prep1' 中 \verb'sort' 的参数非常古怪, 我们可以通过格式化输出, 使得
 \verb'sort' 不再需要任何参数, 然后再让格式化程序对行重新格式化即可.
默认情况下, \verb'sort' 对输入数据按照字母顺序进行排列, 但是在最终的报表
中, 输出首先按照大洲的字母顺序排列, 然后再按人口密度的逆序排列. 为了避免
让  \verb'sort' 带上参数, 准备程序可以预先在每一行的开始处放置一个数,
数的大小依赖于大洲的字母顺序与人口密度, 使得按照这个数进行排序时, 排序结果
是正确的. 数的一种可能的表示方法是大洲的名字, 后面再跟着人口密度的倒数,
见程序 \verb'prep2':
\begin{awkcode}
    # prep2 - prepare countries by continent, inverse pop. den.

    BEGIN { FS = "\t"}
          { den = 1000*$3/$2
            printf("%-15s:%12.8f:%s:%d:%d:%.1f\n",
                $4, 1/den, $1, $3, $2, den) | "sort"
          }
\end{awkcode}
当 \filename{countries} 作为输入时, \verb'prep2' 的输出是:
\begin{shell}
    Asia           :  0.00120000:Japan:120:144:833.3
    Asia           :  0.00169839:India:746:1267:588.8
    Asia           :  0.00359012:China:1032:3705:278.5
    Asia           :  0.03145091:USSR:275:8649:31.8
    Europe         :  0.00157377:Germany:61:96:635.4
    Europe         :  0.00167857:England:56:94:595.7
    Europe         :  0.00383636:France:55:211:260.7
    North America  :  0.00976923:Mexico:78:762:102.4
    North America  :  0.01525316:USA:237:3615:65.6
    North America  :  0.15408000:Canada:25:3852:6.5
    South America  :  0.02452239:Brazil:134:3286:40.8
\end{shell}
格式 \verb'%-15s' 对大洲名来说已经足够宽了, \verb'%12.8f' 对人口密度的倒数
来说, 覆盖范围也已足够. 最终的格式化程序类似于 \verb'form1', 但是忽略了第
2 个字段. 为了简化排序程序的选项, 而特地制造一个排序键, 这种技巧非常常见,
我们会在第 \ref{chap:processing_words} 章的索引程序中再次用到.
\marginpar{92}

如果我们想要一个更加精美的输出, 其只打印大洲名字一次, 那么我们可以使用
程序 \verb'form2':
\begin{verbatim}
    # form2 - format countries by continent, pop. den.

    BEGIN { FS = ":"
            printf("%-15s %-10s %10s %7s %12s\n",
                "CONTINENT", "COUNTRY", "POPULATION",
                "AREA", "POP. DEN.")
          }
          { if ($1 != prev) {
                print ""
                prev = $1
            } else
                $1 = ""
            printf("%-15s %-10s %7d %10d %10.1f\n",
                $1, $2, $3, $4, $5)
          }
\end{verbatim}
执行程序的命令行是 
\begin{verbatim}
    awk -f prep1 countries | awk -f form2
\end{verbatim}
程序的输出是
\begin{verbatim}
    CONTINENT       COUNTRY    POPULATION    AREA    POP. DEN.

    Asia            Japan          120        144      833.3
                    India          746       1267      588.8
                    China         1032       3705      278.5
                    USSR           275       8649       31.8

    Europe          Germany         61         96      635.4
                    England         56         94      595.7
                    France          55        211      260.7

    North America   Mexico          78        762      102.4
                    USA            237       3615       65.6
                    Canada          25       3852        6.5

    South America   Brazil         134       3286       40.8
\end{verbatim}

格式化程序 \verb'form2' 是一个 ``control-break'' 程序, 变量 \verb'prev' 
跟踪大洲的名字, 只有当大洲名字变化时才会打印出来. 在下一节, 我们将会看到 
更复杂的 ``control-break'' 程序.

\subsection{更加复杂的报表}
\label{subsec:a_more_complex_report}

典型的商业报表比我们现在看到的具有更多的内容 (至少在形式上), 为了详细说明,
假设我们需要为每一个大洲作一个小计, 以及计算每一个国家占总人口与总面积的
比重, 我们需要新增一个标题, 以及更多的列表头:
\marginpar{93}
\begin{shell}
\end{shell}

我们仍然可以使用 prepare-sort-format 三步骤策略来生成这张报表, \verb'prep3'
从文件 \filename{countries} 中准备并排序必要的信息:
\begin{shell}
    # prep3 - prepare countries data for form3

    BEGIN  { FS = "\t" }
    pass == 1 {
        area[$4] += $2
        areatot += $2
        pop[$4] += $3
        poptot += $3
    }
    pass == 2 {
        den = 1000*$3/$2
        printf("%s:%s:%s:%f:%d:%f:%f:%d:%d\n",
            $4, $1, $3, 100*$3/poptot, $2, 100*$2/areatot,
            den, pop[$4], area[$4]) | "sort -t: +0 -1 +6rn"
    }
\end{shell}
这个程序需要遍历输入数据两次, 第一次遍历累加每个大洲的面积与人口数, 并分别
保存到数组 \verb'area' 与 \verb'pop' 中, 同时计算总面积与总人口数, 分别
保存在变量 \verb'areatot' 与 \verb'poptot' 中. 第二次遍历对每个国家的统计
结果进行格式化, 并输送给 \verb'sort'. 
\marginpar{94}
两次遍历通过变量 \verb'pass' 控制, 其值可以通过命令行设置:
\begin{awkcode}
    awk -f prep3 pass=1 countries pass=2 countries
\end{awkcode}
