\chapter{报表与数据库}
\label{reports_and_databases}

\marginpar{89}
这章展示如何使用 awk 从文件中提取信息, 并生成报表, 我们把重点放在表格数据,
但是同样的技术也可以用在更加复杂的输入格式上. 本章的主题是开发一个可以与
其他程序配合使用的程序, 我们将会看到大量的, 工作中会经常遇到的数据处理问题,
这些问题很难一步解决, 但是如果多次遍历数据, 就相对比较容易一些.

本章的第一部分讨论如何扫描单个文件来生成报表, 虽然报表的最终格式的确需要花点
心思, 但是其实扫描步骤也是挺复杂的. 第二部分讨论如何从多个相关的文件中
收集数据, 我们考虑用一种比较一般的方法来解决这个问题, 通过把文件组当成
关系数据库, 这样做的好外是字段可以用名字来标识, 而不是数字.

\section{报表生成}
\label{sec:generating_reports}

Awk 可以从文件中挑选数据, 并将挑选到的数据格式化成报表. 我们将使用一个
三步骤过程来生成报表: 准备, 排序, 格式化. 准备步骤包括选择数据, 可能的话
还会对数据进行一些运算, 进而得到期望的信息; 如果我们想让数据按照某种特定
顺序排列, 就必须使用排序步骤, 排序操作可以通过将准备阶段的输出输送给系统
的排序命令来完成; 格式化操作由第 2 个 awk 程序来完成, 它根据已排序的数据
生成报表. 为了详细说明, 在这一节 我们利用第 \ref{chap:the_awk_language} 
章的文件\filename{countries} 来生成几张报表.

\subsection{一个简单的报表}
\label{subsec:a_simple_report}

假设我们想要一张报表, 这张表包含了每个国家的人口, 面积, 及人口密度. 
我们还希望国家按照所在的大洲进行分组, 大洲按照字母顺序排列, 在每个大洲里,
国家按照人口密度的降序排列, 就像这样:
\marginpar{90}
