% vim: ts=8 sts=8 sw=4 et tw=75
\chapter{后记}
\label{chap:epilog}
\marginpar{181}

能看到这里, 说明读者在某种程度上已经是一个熟练的 awk 用户了, 至少不再是
一个笨拙的初学者. 当你在学习书中的示例程序时, 以及自己写程序的过程中, 
可能想知道 awk 为什么会是现在这个样子, 是否还有需要改进的地方.

本章的第一部分先讲一些历史故事, 然后讨论一下作为编程语言使用时, awk 有
哪些优点和缺点. 第二部分探讨 awk 程序的性能, 另外, 如果某个问题过于庞大,
以致于无法用一个单独的程序来解决时, 文章也提供了一些对问题进行重新规划
的方法.

\section{作为语言的 AWK}
\label{sec:awk_as_a_language}

关于 awk 的工作开始于 1977 年. 在那时候, 搜索文件的 Unix 程序
(\texttt{grep} 和 \texttt{sed}) 只支持正则表达式模式, 并且唯一能做的动作
只能替换和打印一整行数据, 还不存在字段和非数值操作. 我们当时的目标是
开发一款模式识别语言, 该语言支持字段, 包括用模式来匹配字段, 以及用动作
来操作字段. 最开始, 我们只能想它来转换数据, 验证程序的输入, 通过处理
输出数据来生成报表, 或对它们重新编排, 来作为其他程序的输入.

1977 年的 awk 只有很少的内建变量和预定义函数, 当时只是用它来写一些很简短
的程序, 就像第 \ref{chap:an_awk_tutorial} 章中出现的那些小程序. 后来,
我们写了一个小教程, 来指导新来的同事如何使用 awk. 正则表达式的表示法
来源于 \texttt{lex} 和 \texttt{egrep}, 其他的表达式和语句则来源于 C
语言.

Out model was that an invocation would be one or two lines long, typed in
and used immediately. 默认操作都是为了满足这个要求, 具体来说, 使用空格
作为默认的字段分隔符, 隐式地初始化, 变量的无类型声明, 等等, 这些都使得
单行程序变成可能. 作为作者, 我们非常清楚地知道 awk 将会被如何地使用,
所以我们只写单行程序.\footnote{原文是 We, being the authors, ``knew''
    how the language was supposed to be used, and so we only wrote
one-liners.}

\marginpar{182}
Awk 的快速传播强有力地推动了语言的发展. 把 awk 作为一门通用编程语言来使用,
而且能够这么快速地流行起来, 我们都感到非常的惊喜, 当看到一个无法在
一页内显示完毕的 awk 程序时, 我们的第一反应是震惊和惊异. 之所以会出现这种
情况是因为许多人在使用计算机时, 仅限于 shell (命令行语言) 和 awk, 而不
是使用一门 ``真正'' 的编程语言来开发程序 --- 他们经常过度伸展他们所喜爱
的工具.
