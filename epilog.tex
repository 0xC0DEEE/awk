% vim: ts=8 sts=8 sw=4 et tw=75
\chapter{后记}
\label{chap:epilog}
\marginpar{181}

能看到这里, 说明读者在某种程度上已经是一个熟练的 awk 用户了, 至少不再是
一个笨拙的初学者. 当你在学习书中的示例程序时, 以及自己写程序的过程中, 
可能想知道 awk 为什么会是现在这个样子, 是否还有需要改进的地方.

本章的第一部分先讲一些历史故事, 然后讨论一下作为编程语言使用时, awk 有
哪些优点和缺点. 第二部分探讨 awk 程序的性能, 另外, 如果某个问题过于庞大,
以致于无法用一个单独的程序来解决时, 文章也提供了一些对问题进行重新规划
的方法.

\section{作为语言的 AWK}
\label{sec:awk_as_a_language}

关于 awk 的工作开始于 1977 年. 在那时候, 搜索文件的 Unix 程序
(\texttt{grep} 和 \texttt{sed}) 只支持正则表达式模式, 并且唯一能做的动作
只有替换和打印一整行数据, 还不存在字段和非数值操作. 我们当时的目标是
开发一款模式识别语言, 该语言支持字段, 包括用模式来匹配字段, 以及用动作
来操作字段. 最初, 我们只能想用它来转换数据, 验证程序的输入, 通过处理
输出数据来生成报表, 或对它们重新编排, 来作为其他程序的输入.

1977 年的 awk 只有很少的内建变量和预定义函数, 当时只是用它来写一些很简短
的程序, 就像第 \ref{chap:an_awk_tutorial} 章中出现的那些小程序. 后来,
我们写了一个小教程, 来指导新来的同事如何使用 awk. 正则表达式的表示法
来源于 \texttt{lex} 和 \texttt{egrep}, 其他的表达式和语句则来源于 C
语言.

我们希望程序能够尽量得简洁, 最好只有一两行, 这样就能够快速地输入并执行,
\footnote{ 原文是 Our model was that an invocation would be one or two
    lines long, typed in and used immediately.}
    默认操作都是为了向这个方向努力, 具体来说, 使用空格
作为默认的字段分隔符, 隐式地初始化, 变量的无类型声明, 等等, 这些设计
选择都使得 单行程序变成可能. 作为作者, 我们非常清楚地 ``知道'' awk
将会被如何地使用, 所以我们通常只需要写单行程序就够了.\footnote{原文是 We,
    being the authors, ``knew'' how the language was supposed to be used,
    and so we only wrote one-liners.}

\marginpar{182}
Awk 的快速传播强有力地推动了语言的发展. 把 awk 作为一门通用编程语言来使用,
而且能够这么快速地流行起来, 我们都感到非常的惊喜, 当看到一个无法在
一页内显示完毕的 awk 程序时, 我们的第一反应是震惊和惊异. 之所以会出现这种
情况是因为许多人在使用计算机时, 仅限于 shell (命令行语言) 和 awk, 而不
是使用一门 ``真正'' 的编程语言来开发程序 --- 他们经常过度伸展所喜爱
的工具.

为变量的值同时维护两种表示形式: 字符串与数值, 根据上下文来选择合适的形式
--- 这只是一个实验性设计, 目的是为了尽可能地使用同一套运算符集合来写出
简短的程序, 在字符串与数值的界限很模糊的情况下, 程序也要能正确地工作.
最终目标完成地很好, 但偶尔也会因为粗心而得到意料之外的运行结果. 第 
\ref{chap:the_awk_language} 章介绍的规则可以用来解决界限模糊的情况, 它
们都来源于用户的使用经验.

关联数组的灵感来源于 SNOBOL4 表格 (虽然它们不具备 SNOBOL4 表格的通用性).
诞生 awk 的机器内存很小, 而且速度很慢, 正是这个环境造就了数组现在的性质.
把下标类型限制为字符串是其中一种表现, 另外的限制还包括单维数组 (虽然套了
一层语法外衣, 但本质上还是一维数组). 一个更加通用的实现应该支持多维数组,
至少支持数组元素可以是另外一个数组.

Awk 的主要设施在 1985 年被加入进来, 主要是为了满足用户需求. 添加的
功能包括动态正则表达式, 新的内建变量与内建函数, 多输入流, 以最重要的用
户自定义函数.

\texttt{match}, 动态正则表达式和新的字符串替换函数提供了非常有用的功能,
而且对用户来说, 只是稍微增加了一点复杂度.

在 \texttt{getline} 被引入之前, 输入数据的唯一种类是 \patact 语句所隐含
着的隐式输入循环. 这个限制条件确实太强了. 在原来的 awk 版本中, 对于具有
多个输入数据源的程序 (比如套用信函生成程序) 来说, 必须通过设置一个标志变量
(或其他类似的技巧) 来读取数据源. 而在新版的 awk 中, 多个输入数据可以在
\texttt{BEGIN} 部分, 用 \texttt{getline} 来读取. 另一方面, \texttt{getline}
是过载的, 它的语法和其他表达式相比并不一致. 其中一个问题是 \texttt{getline}
需要返回它所读取到的数据, 但同时也会返回表示成功或失败的返回值.

用户自定义函数的实现是一个折中方案, 从 awk 的最初设计开始, 出现了许多
困难. 我们并不打算在语言中加入声明, 这个设计造成的一个结果是声明局部
变量的特殊方法 --- 把局部变量写到参数列表中. 这种做法不仅看起来很陌生,
而且会让大型程序更容易出错. 另外, 缺少显式的字符串拼接运算符可以让程序
更加简短, 但这同时也要求在调用函数时, 必须在函数名之后紧跟上左括号. 不
管怎么说, 新的特性使得用 awk 编写大型程序变得更加容易.
\marginpar{183}

\section{性能}
\label{sec:performance}
在某种程度上, awk 是很有吸引力的 --- 通常情况下, 用它来编写你所需要的程序
非常容易, 而且在面对适当规模的数据时, 处理起来也足够快, 特别是在程序本身
也会变化的情况下.

然而, 当处理的数据规模越来越大时, awk 程序就会越来越慢. 从常理上讲, 这
种现象是很正常的, 但是等待结果的过程常常使人无法忍受. 解决这种问题
的办法都比较复杂, 但是本节提出的一些建议或许能对你产生一些帮助.

当程序的运行时间过长时, 除了忍耐, 可以试着从其他几个方面入手. 首先, 让
程序运行得更快是可能的 --- 或者利用更好的算法, 或者是把频繁执行的操作,
用等价的, 但是更轻量的操作替换掉. 在第
\ref{chap:experiments_with_algorithms} 章你已经见到了一个优秀的算法能够
产生的巨大作用 --- 即使是在数据规模只有适度增加的情况下, 线性算法和平方
算法之间也会产生巨大的差距. 然后, 你可以限制 awk 程序的功能, 而使用其
他更快速的程序和 awk 配合. 最后, 你也可以用其他编程语言重写程序.

在你着手提高程序的性能之前, 你必须知道程序的时间都花在了哪里. 即使是在 
每种操作和底层硬件非常接近的编程语言中, 人们对时间开销的分布所作出的估计
也会非常得不可靠.
