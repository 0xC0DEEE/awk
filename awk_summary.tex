% vim: ts=8 sts=8 sw=4 et tw=75
\chapter{AWK 总结}
\label{chap:awk_summary}

\marginpar{187}
这个附录包含了 awk 语言的一个总结. 在句法规则上, 如果某个成分被一对中括号
\verb'['...\verb']' 包围, 则表示它们是可选的.

\subsubsection{命令行}
\begin{quote}
    \texttt{awk [-F}\textit{s}\texttt{] '}%
    \textit{program}\texttt{'}\ \ \textit{optional list of filenames}

    \texttt{awk [-F}\textit{s}\texttt{] -f}
    \textit{progfile}\ \ \textit{optional list of filenames}
\end{quote}
参数 \texttt{-F}\textit{s} 把字段分隔符 \texttt{FS} 设置成 \textit{s},
如果没有提供文件名, awk 就从标准输入读取数据. 文件名的形式可以是 
\textit{var}\texttt{=}\textit{text}, 在这种情况下, 相当于把 \textit{text}
赋值给变量 \textit{var}, 当这个参数被当作一个文件而被访问时, 执行赋值
操作.

\subsubsection{AWK 程序}
一个 awk 程序由一系列的 \patact 语句和函数定义组成. 一个 \patact 语句具有
形式:
\begin{quote}
    \textit{pattern}\ \ \verb'{'\ \textit{action}\ \verb'}'
\end{quote}
如果某个动作省略了模式, 则默认匹配所有输入行; 如果某个模式省略了动作, 则
默认打印匹配行.

一个函数定义具有形式:
\begin{quote}
    \texttt{function}\ \
    \textit{name}\verb'('\textit{parameter-list}\verb') {'
        \textit{statement}\ \verb'}'
\end{quote}
\patact 语句和函数定义由换行符或分号分隔, 并且这两个字符可以混合使用.

\subsubsection{模式}
\begin{quote}
    \texttt{BEGIN}

    \texttt{END}

    \textit{expression}

    \verb'/'\textit{regular expression}\verb'/'

    \textit{pattern}\ \verb'&&'\ \textit{pattern}

    \textit{pattern}\ \verb'||'\ \textit{pattern}

    \verb'!'\textit{pattern}

    \verb'('\textit{pattern}\verb')'

    \textit{pattern}\verb','\ \textit{pattern}
\end{quote}

最后一个模式是范围模式, 它不能作为其他模式的组成部分. 类似地,
\texttt{BEGIN} 和 \texttt{END} 也不能和其他模式结合.
\marginpar{188}
\subsubsection{动作}
一个动作由一系列的语句组成, 这些语句包括:
\begin{quote}
    \texttt{break}

    \texttt{continue}

    \texttt{delete}\ \textit{array-element}

    \texttt{do}\ \textit{statement}\ \texttt{while}\
    \texttt{(}\textit{expression}\texttt{)}

    \texttt{exit[}\textit{expression}\texttt{]}

    \textit{expression}

    \texttt{if (}\textit{expression}\texttt{) }\textit{statement}
    \ \texttt{[else}\ \textit{statement}\texttt{]}

    \textit{input-output statement}

    \texttt{for (}\textit{expression}\texttt{; }\textit{expression}
    \texttt{; }\textit{expression}\texttt{) }\textit{statement}

    \texttt{for (}\textit{variable}\texttt{ in }\textit{array}\texttt{) }
    \textit{statement}

    \texttt{next}

    \texttt{return [}\textit{expression}\texttt{]}

    \texttt{while (}\textit{expression}\texttt{) statement}

    \verb'{' \textit{statement} \verb'}'
\end{quote}
一个单独的分号表示空语句. 在一个 \texttt{if-else} 语句中, 如果第一个 
\textit{statement} 和 \texttt{else} 出现在同一行, 那么它必须以分号结尾, 
或者用花括号包围起来. 类似地, 在 \texttt{do} 语句中, 如果
\textit{statement} 和 \texttt{while} 出现在同一行, 那么它必须以分号结尾,
或者用花括号包围起来.

\subsubsection{程序格式}
语句通过换行符或 (和) 分号隔开. 空行可以出现在任何语句, \patact 语句,
或函数定义的前面或后面. 空格与制表符可以插入到运算符或操作数的周围. 一条
长语句可以通过反斜杆延续到下一行. 另外, 如果一条语句在逗号, 左花括号,
\verb'&&', \verb'||', \texttt{do}, \texttt{else}, \texttt{if} 或 
\texttt{for} 的右括号后断行, 则不需要反斜杆. 由 \verb'#' 开始的注释可以
出现在任意一行的末尾.

\subsubsection{输入输出}
\begin{quote}
    \begin{tabbing}
        \texttt{close(}\textit{expr}\texttt{)}  \= 关闭由 \textit{expr}
        指示的文件或管道 \\
        \texttt{getline} \> 把 \verb'$0' 设置成下一条记录; 同时设置
        \texttt{NF}, \texttt{NR}, \texttt{FNR} \\

        \texttt{getline <}\textit{file} \> 把 \verb'$0' 设置成文件
        \textit{file} 的下一条记录; 同时设置 \texttt{NF} \\

        \texttt{getline}\ \textit{var} \> 把 \textit{var} 设置成下一条记录;
        同时设置 \texttt{NR}, \texttt{FNR} \\

        \texttt{getline}\ \textit{var}\ \texttt{<}\textit{file} \>  把
        \textit{var} 设置成文件  \textit{file} 的下一条记录. \\

        \texttt{print}  \> 打印当前记录 \\

        \texttt{print}\ \textit{expr-list} \> 打印 \textit{expr-list}
        所表示的表达式 \\

        \texttt{print}\ \textit{expr-list}\ \texttt{>}\textit{file} \>
        把表达式输出到文件 \textit{file} 中 \\

        \texttt{printf}\ \textit{fmt}\texttt{,} \textit{expr-list} \>
        格式化并输出 \\

        \texttt{printf}\ \textit{fmt}\texttt{,} \textit{expr-list}\ 
        \texttt{>} \textit{file}        \> 格式化并输出到文件 \textit{file}
        中 \\

        \texttt{system(}\textit{cmd-line}\texttt{)}     \> 执行命令
        \textit{cmd-line}, 返回命令的退出状态 \\
    \end{tabbing}
\end{quote}

\texttt{print} 后面的 \textit{expr-list}, 以及 \texttt{printf} 后面的
\textit{fmt}\texttt{,}\ \textit{expr-list} 可以用括号括起来. 在 
\texttt{print} 和 \texttt{printf} 中, \texttt{>>}\textit{file} 表示把
输出追加到文件 \textit{file} 的末尾, \texttt{|}\ \textit{command} 表示把
输出写到一个管道中. 类似的, \textit{command}\ \texttt{| getline} 表示把
命令 \textit{command} 的输出以管道的方式输送给 \texttt{getline}. 函数
\texttt{getline} 在遇到文件末尾时返回 0, 出错时返回 -1.

\marginpar{189}
