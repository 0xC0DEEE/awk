\chapter{文本处理}
\label{chap:processing_words}

\marginpar{111}
本章的程序指向一个共同主题: 文本处理. 示例程序涵盖的范围包括随机单词
与句子的生成 (生成的句子可以和用户进行有限的对话), 以及文本处理. 大多数示例
程序都很简单, 它们只是起说明作用, 但是, 其中一些文档准备程序的确拥有实际
用途.

\section{随机文件发生器}
\label{sec:random_text_generation}

生成随机数据的程序有许多种用途. 这种程序可以用内建函数 \texttt{rand} 来
创建, 该函数每次被调用时, 都会返回一个伪随机数. \texttt{rand} 每次都使用
同一个种子数来生成随机数, 所以, 如果你想要得到一个不同的随机数序列, 你
就必须调用一次 \texttt{srand()}, 函数根据当前时间计算出一个种子数, 并用
该种子数初始化 \texttt{rand}.

\subsection{随机选择}
\label{subsec:random_choices}

\texttt{rand} 每次被调用时都会返回 一个大于等于 0, 小于 1 的浮点数, 但是
一般来说, 更通常的需求是返回一个 \texttt{1} 到 \texttt{n} 之间的随机整数,
我们可以用 \texttt{rand} 来实现:
\begin{awkcode}
    # randint - return random integer x, 1 <= x <= n

    function randint(n) {
        return int(n * rand()) + 1
    }
\end{awkcode}
\texttt{randint(n)} 按比例调整 \texttt{rand} 的返回值, 调用整后的值大于
等于 \texttt{0} 并且小于 \texttt{n}, 将小数部分截去可以得到 \texttt{0} 
到 \texttt{n-1} 的整数, 然后再加 1, 就是 \texttt{1} 到 \texttt{n} 之间的 
整数.

我们可以用 \texttt{randint} 来随机选择一个字母:
\marginpar{112}
\begin{awkcode}
    # randlet - generate random lower-case letter

    function randlet() {
        return substr("abcdefghijklmnopqrstuvwxyz", randint(26), 1)
    }
\end{awkcode}
