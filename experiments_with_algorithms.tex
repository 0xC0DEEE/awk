% vim: ts=8 sts=8 sw=4 et tw=75
\chapter{算法实验}
\label{chap:experiments_with_algorithms}
\marginpar{153}

一般而言, 理解事物如何工作的最好方式就是自己动手做一些小实验, 算法学习
就是一个典型的例子: 编写实际代码有助于弄清楚那些容易被伪码掩盖的问题.
不仅如此, 最终得到的程序是运行的, 通过观察运行结果, 就可以知道算法的
的正确性, 而这是伪码所无法办到的.

Awk 很适合做这种测试工作. 如果某个程序使用 awk 编写, 那我们就可以把精力
集中在算法上, 而不是语言本身. 如果某个算法最终要应用到某个大型程序中,
那么先让算法能够单独地运行起来可能会更有效率. 当需要为某个算法进行调试,
测试与性能评价时, 通常需要构造一些脚手架程序, 在这一方面, awk 是一款
优秀的脚手架构造工具, 它并不管算法本身是用什么语言实现的.

这一章讨论算法实验. 前半章描述三种排序算法, 这三种算法常常是算法课首
先要介绍的内容, 我们将使用 awk 程序对这些算法进行测试, 性能度量和
刻画. 后半章展示几种拓扑排序算法, 实现 Unix 的文件更新实用程序
\texttt{make}.

\section{排序}
\label{sec:sorting}

这一小节讨论三种著名并且很有用的算法: 插入排序, 快速排序, 以及堆排序.
插入排序非常简单, 但是只有在元素很少的情况下, 效率才足够高; 快速排序
是最好的通用排序算法之一; 堆排序可以保证即使在最坏的情况下, 也可以拥有
较高的效率. 我们对每一种算法都进行介绍, 并加以实现, 然后再用测试例程
对它们进行测试, 最后评价性能.

\subsection{插入排序}
\label{subsec:insertion_sort}
\cterm{基本概念}. 插入排序的过程类似于给一堆卡片排序: 每次从卡片堆里
拿出一张, 把它插入到合适的位置.\footnote{原文为 Insertion sort is similar
    to the method of sorting a sequence of cards by picking up the cards
    one at a time and inserting each card into its proper position in the
hand}
\marginpar{154}
