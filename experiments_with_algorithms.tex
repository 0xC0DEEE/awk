% vim: ts=8 sts=8 sw=4 et tw=75
\chapter{算法实验}
\label{chap:experiments_with_algorithms}
\marginpar{153}

一般而言, 理解事物如何工作的最好方式就是自己动手做一些小实验, 算法学习
就是一个典型的例子: 编写实际代码有助于弄清楚那些容易被伪码掩盖的问题.
不仅如此, 最终得到的程序是可运行的, 通过观察运行结果, 就可以知道算法的
的正确性, 而这是伪码所无法办到的.

Awk 很适合做这种测试工作. 如果某个程序使用 awk 编写, 那我们就可以把精力
集中在算法上, 而不是语言本身. 如果某个算法最终要应用到某个大型程序中,
那么先让算法能够单独地运行起来可能会更有效率. 当需要为某个算法进行调试,
测试与性能评价时, 通常需要构造一些脚手架程序, 在这一方面, awk 是一款
优秀的脚手架构造工具, 它并不关心算法本身是用什么语言实现的.

这一章讨论算法实验. 前半章描述三种排序算法, 这三种算法常常是算法课首
先要介绍的内容, 我们将使用 awk 程序对这些算法进行测试, 性能度量和
刻画. 后半章展示几种拓扑排序算法, 实现 Unix 的文件更新实用程序
\texttt{make}.

\section{排序}
\label{sec:sorting}

这一小节讨论三种著名并且很有用的算法: 插入排序, 快速排序, 以及堆排序.
插入排序非常简单, 但是只有在元素很少的情况下, 效率才足够高; 快速排序
是最好的通用排序算法之一; 堆排序可以保证即使在最坏的情况下, 也可以拥有
较高的效率. 我们对每一种算法都进行介绍, 并加以实现, 然后再用测试例程
对它们进行测试, 最后评价性能.

\subsection{插入排序}
\label{subsec:insertion_sort}
\cterm{基本概念}. 插入排序的过程类似于给一堆卡片排序: 每次从卡片堆里
拿出一张, 把它插入到合适的位置.\footnote{原文为 Insertion sort is similar
    to the method of sorting a sequence of cards by picking up the cards
    one at a time and inserting each card into its proper position in the
hand}

\cterm{实现}. 下面的代码使用插入排序对数组 \texttt{A[1]}, ...,%
\marginpar{154}%
\texttt{A[n]} 进行升序排列. 第一个动作把输入数据读取到一个数组中,
\texttt{END} 动作调用函数 \texttt{isort} 对数组进行排序, 最后输出排序
结果:
\begin{awkcode}
    # insertion sort 
        { A[NR] = $0 }
    END { isort(A, NR)
          for (i = 1; i <= NR; i++)
              print A[i]
        }

    # isort - sort A[1..n] by insertion
    function isort(A, n,    i, j, t) {
        for (i = 2; i <= n; i++)
            for (j = i; j > 1 && A[j-1] > A[j]; j--) {
                # swap A[j-1] and A[j]
                t = A[j-1]; A[j-1] = A[j]; A[j] = t
            }
    }
\end{awkcode}
\texttt{isort} 函数内的外层循环在每次迭代开始时, 数组\texttt{A} 的元素
\texttt{1} 至元素 \textit{i}\texttt{-1} 就已经处于有序状态. 内层循环每
次迭代
都把当前处于第 \textit{i} 个位置上的元素向前移动, 跳过所有比它大的元素. 
当外层循环结束时, 所有的 \textit{n} 个元素都处于有序状态.

数值或字符串都可以用这个程序进行排序. 但是当输入数据同时含有数值与字符
串时, 就要小心一点 --- 由于强制类型转换, 比较结果可能会让你感到惊讶.

如果数组 \texttt{A} 含有以下 8 个整数:
\begin{file}
    8 1 6 3 5 2 4 7
\end{file}
那么排序的过程如下所示:
\begin{file}
    8|1 6 3 5 2 4 7
    1 8|6 3 5 2 4 7
    1 6 8|3 5 2 4 7
    1 3 6 8|5 2 4 7
    1 3 5 6 8|2 4 7
    1 2 3 5 6 8|4 7
    1 2 3 4 5 6 8|7
    1 2 3 4 5 6 7 8|
\end{file}
竖线符把数组的已排序部分和未排序部分分开.

\cterm{测试}. 应该如何测试 \texttt{isort}? 我们可以每次输入一点数据,
并查看排序结果, 当然, 这样做并没有错, 可是对于任意规模的程序来说, 
这种方法不能做到详尽的测试. 第二种方案是自动生成大量的随机数集合, 
把这些集合作为 \texttt{isort} 的输入数据. 这的确是一个不错的办法,
但是还可以做得更好: 为了测试程序的薄弱环节, 我们还需要构造一些特殊
的测试用例, 用来测试边界与异常情况.
\marginpar{155} 对排序来说, 典型的边界与异常情况包括:
\begin{pattern}
\indent 序列长度为 0 \par
\indent 序列长度为 1 \par
\indent 序列包含 \textit{n} 个随机数 \par
\indent 序列包含 \textit{n} 个已排序的数 \par
\indent 序列包含 \textit{n} 个逆序排列的数 \par
\indent 序列包含 \textit{n} 个相同的数
\end{pattern}

本章的目标之一是展示如何使用 awk 来帮助测试和评价程序. 为了说明, 我们现在
要对排序例程的测试与运行结果评价过程进行自动化.

主要有两种办法来实现测试与评价过程的自动化, 每一种都有它各自的优点. 第一 
种称为 ``批处理模式'': write a program to execute a pre-planned set of
tests, exercising the sort function as suggested above.\footnote{TODO}
下面的程序可以生成测试数据并检查测试结果. 除了 \texttt{isort}, 还有其他
几个函数, 它们用来生成不同类型的数组, 以及检查排序后的数组是否是有序的.
\begin{awkcode}
    # batch test of sorting routines

    BEGIN {
        print "    0 elements"
        isort(A, 0); check(A, 0)    
        print "    1 element"
        genid(A, 1); isort(A, 1); check(A, 1)
        
        n = 10
        print "    " n " random integers"
        genrand(A, n); isort(A, n); check(A, n)
        
        print "    " n " sorted integers"
        gensort(A, n); isort(A, n); check(A, n)
        
        print "    " n " reverse-sorted integers"
        genrev(A, n); isort(A, n); check(A, n)
        
        print "    " n " identical integers"
        genid(A, n); isort(A, n); check(A, n)
    }

    function isort(A,n,     i,j,t) {
        for (i = 2; i <= n; i++)
            for (j = i; j > 1 && A[j-1] > A[j]; j--) {
                # swap A[j-1] and A[j]
                t = A[j-1]; A[j-1] = A[j]; A[j] = t
            }
    }

\end{awkcode}
\marginpar{156}
\begin{awkcode}
    # test-generation and sorting routines...

    function check(A,n,   i) {
        for (i = 1; i < n; i++)
            if (A[i] > A[i+1])
                printf("array is not sorted, element %d\n", i)
    }

    function genrand(A,n,  i) { # put n random integers in A
        for (i = 1; i <= n; i++)
            A[i] = int(n*rand())
    }

    function gensort(A,n,  i) { # put n sorted integers in A
        for (i = 1; i <= n; i++)
            A[i] = i
    }

    function genrev(A,n,  i) {  # put n reverse-sorted integers
        for (i = 1; i <= n; i++)  # in A
            A[i] = n+1-i
    }

    function genid(A,n,  i) {   # put n identical integers in A
        for (i = 1; i <= n; i++)
            A[i] = 1
    }
\end{awkcode}

第二种方法相对来说没那么方便, 但很适合用 awk 来处理. 基本思想是构建一个
构架程序, 利用该框架可以很容易以交互性的方式来完成测试. 交互式方案是
批处理模式的一个很好的补充, 特别是当待测试的算法没有排序那么容易理解
时. 交互式处理模式在调试程序时也很方便.

特别地, 我们将要设计的程序, 在效果上等价于一个专门用于构造测试数据与操作
的微型语言, 因为语言并不需要做太多的工作, 也不必处理大量用户的情况,
所以不用设计得多么复杂. 如果必要的话, 我们甚至可以丢掉写了一半的代码,
重新开始. Our language provides for automatic generation of an array, and,
looking ahead to the rest of this chapter, for naming the sort to be
exercised.\footnote{TODO} 我们省略了排序和数据生成子程序, 它们和前一个
示例相同.

程序的基本组织是一系列的正则表达式, 它们负责扫描输入数据, 判断数据类型
和使用的排序算法. 如果某个输入数据不与任何一个模式相匹配, 程序就会输出
一条错误消息, 并演示正确的使用方法. 如果仅仅说明输入数据有错, 可能没
多大帮助.

\marginpar{157}
\begin{awkcode}
    # interactive test framework for sort routines

    /^[0-9]+.*rand/ { n = $1; genrand(A, n); dump(A, n); next }
    /^[0-9]+.*id/   { n = $1; genid(A, n); dump(A, n); next }
    /^[0-9]+.*sort/ { n = $1; gensort(A, n); dump(A, n); next }
    /^[0-9]+.*rev/  { n = $1; genrev(A, n); dump(A, n); next }
    /^data/ {   # use data right from this line
            for (i = 2; i <= NF; i++)
                    A[i-1] = $i
            n = NF - 1
            next
    }
    /q.*sort/ { qsort(A, 1, n); check(A, n); dump(A, n); next }
    /h.*sort/ { hsort(A, n); check(A, n); dump(A, n); next }
    /i.*sort/ { isort(A, n); check(A, n); dump(A, n); next }
    /./  { print "data ... | N [rand|id|sort|rev]; [qhi]sort" }

    function dump(A, n) {    # print A[1]..A[n]
            for (i = 1; i <= n; i++)
                    printf(" %s", A[i])
            printf("\n")
    }

    # test-generation and sorting routines ...
    ...
\end{awkcode}
正则表达式提供了一种非常宽松的输入语法: 只要短语和快速排序稍微有点接近,
比如 ``quicksort'', 就选择快速排序算法. 我们也可以直接手工输入数据, 
而不是自动生成, 这个功能允许我们既可以基于文本, 也可以基于数字对算法
进行测试. 为了说明, 上面程序的一个输出是:
\begin{shell}
    10 random 
     9 8 4 6 7 2 4 0 4 0
    isort 
     0 0 2 4 4 4 6 7 8 9
    10 reverse 
     10 9 8 7 6 5 4 3 2 1
    qsort
     1 2 3 4 5 6 7 8 9 10 
    data now is the time for all good men
    hsort
     all for good is men now the time
\end{shell}

\cterm{性能}. \texttt{isort} 所执行的操作次数取决于 \textit{n} 的值,
即待排序的元素个数, 以及它们原来的排列顺序. 插入排序是 \cterm{平方} 级
算法, 也就是说, 在最坏的情况下, 算法的运行时间增长速度与元素个数的平方成
正比. 这意味着如果元素个数变成原来的 2 倍, 那么运行时间将会是原来的
4 倍. 如果待排序元素碰巧就处于一种基本有序的状态, 那么程序的工作量
就会少很多, 于是运行时间将会按照线性增长, 线性增长指的是与元素的个数
成正比.
